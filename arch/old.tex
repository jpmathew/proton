\documentclass[a4paper,10pt,fleqn,titlepage,twoside]{article}
\usepackage{amsmath}
\usepackage{graphicx}
\usepackage[colorlinks,linkcolor=red,urlcolor=red,citecolor=red,plainpages=false,pdfpagelabels,breaklinks]{hyperref}
\pagestyle{plain}

\title{A Few Thoughts on A2D conversion}
\author{
        Joseph Palackal Mathew \\
	Department of Electrical Engineering\\
        University of California,Los Angeles\\ 
	\href{mailto:jpmathew@ucla.edi}{jpmathew@ucla.edu}
}
\date{\today}
\begin{document}
\maketitle

\section*{Problem Definition}
Aim is to arrive at a topology that will minimize power While meeting target spec of high resolution and high speed.

\newpage

\section*{Problem Analysis}

We are given an input $x\epsilon[-1,1]$ and the problem invloves finding K such that $$-1+K\Delta \leq x \leq -1+(K+1)\Delta$$ If the input
is random with uniform distribution then the problem will require a minimum $N$ ``Yes or No'' questions with precise where $$N=log_2(2/\Delta)$$
answers. From a circuit perspective ``Yes or No' is that of a comparator comparing input with a threshold with infinite precision. If the
comparison has finite precision i.e it is unreliable for $|inp| < \epsilon $ then at the end of $N$ questions input can be as far as
$\epsilon$ away from the said range.

A SAR A2D conversion proceeds as follows . We compare the input with a threshold $$V_{th}(k) = 0.5*(x_{max}(k)+x_{min}(k)) $$ where it's
guranteed by previous knowledge that $x \epsilon [x_{max}(k),x_{min}(k)]$.If the comparison is precise then input is guaranteed to be in
$[x_{max}(k),V_{th}(k)]$ or $[x_{min}(k),V_{th}(k)]$ so by setting $x_{max}(k+1)$ and $x_{min}(k+1)$ accordingly we can gurantee that
$S(k+1)=x_{max}(k+1)-x_{min}(k+1) = 0.5(x_{max}(k)-x_{min}(k))$ and new $V_{th}(k+1) =(1 \pm 0.5 )*V_{th}(k)$ . Given initial condition
on input we will get $S(k)=2/2^{k-1}$ and movement in threshold after $k^{th}$ comparison as $1/2^k$

There are three types of errors which manifest as $\epsilon$ in a comparator namely offset,noise and settling errors.



\end{document}
