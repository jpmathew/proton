\documentclass[a4paper,10pt,fleqn,titlepage,twoside]{article}
\usepackage{amsmath}
\usepackage{graphicx}
\usepackage[colorlinks,linkcolor=red,urlcolor=red,citecolor=red,plainpages=false,pdfpagelabels,breaklinks]{hyperref}
\pagestyle{plain}

\title{Architecture for a High Speed SAR ADC}
\author{
        Joseph Palackal Mathew \\
	Department of Electrical Engineering\\
        University of California,Los Angeles\\ 
	\href{mailto:jpmathew@ucla.edi}{jpmathew@ucla.edu}
}
\date{\today}
\begin{document}
\maketitle

\section*{Problem Definition}
Derive a topology that will minimize power while achiveing 12 bits of resolution and 200MSPS speed . Make topology amenable for future interleaving 
that can improve speed.

\section*{Problem Analysis}

We are given an input $x\:\epsilon\:[-1,1]$ and converter tries to find K such that $$-1+K\Delta \leq x \leq -1+(K+1)\Delta.$$ If input
is uniformly distributed then the problem will require a minimum $N=log_2(2/\Delta)$ ``Yes or No'' questions with precise answers. From a circuit
perspective ``Yes or No' is that of a comparator comparing input with a threshold with infinite precision. If the
comparison has finite precision i.e it is unreliable for $|inp| < \epsilon $ then at the end of $N$ questions input can be as far as
$\epsilon$ away from the said range.

A typical SAR ADC conversion proceeds as follows . We compare the input with a threshold $V_{th}(k) = r(k)*(x_{max}(k)+x_{min}(k)) + \epsilon(k) $ where it's
guranteed by previous knowledge that $x\:\epsilon\:[x_{max}(k),x_{min}(k)]$.At the end of comparison the input is guaranteed to be in
$[x_{max}(k),V_{th}(k)]$ or $[x_{min}(k),V_{th}(k)]$ so by setting $x_{max}(k+1)$ and $x_{min}(k+1)$ accordingly we can gurantee that range in which 
input can lie $S(k+1)=x_{max}(k+1)-x_{min}(k+1) < S(k)$ provided that we put the thresholds  $V_{th}(k)\:\epsilon[x_{max}(k),x_{min}(k)]$ . This by iteration 
can guarantee that we can locate input to a finite precision in required number of steps.
\newpage
\subsection*{Sampling Circuit}
Noise power due to Differential sampling referred to input
$$P_{noise} = 2*\frac{KT}{Cs}*\frac{Cs+Cp}{Cs}$$
where $Cs$ = sampling cap and $Cp$ = parasitic cap on top-plate , assuming top plate switch is limiting bandwidth and contributing most of noise compared to bottom plate switch . This is a reasonable 
assumption as every attempt will be made to minimize the top plate switch to reduce charge injection offset.
Input referred SNR assuming differential swing of $\pm V_{REF}$
$$SNR = {V_{REF}}^{2}/2*P_{noise}$$
Assuming a shrink of .8 and $V_{REF} = 0.9$,for a noise performance equaling 12 bit quantization noise $$P_{noise}=\Delta^2/12$$$$Cs=600fF$$
\newpage
\subsection{Clocked Comparator}
\paragraph{}
	In a latched comparator $V_{in}*e^{t/\tau}=VDD$ . For a minimum resolution input $V_{\delta}$ it will need a time $T_{\delta}  = \tau*ln(VDD/V_{\delta})
$.Input greater than $V_{\delta}$ needs time less time to resolve than $T_{\delta}$.An asynchronous ADC uses this fact for optimization by making every comparison time just enough to make
accurate decision by checking output differential level against VDD or timing out after $T_{max}$. Hence $$T_{cmp}=\tau*ln(VDD/V_{in}).$$ This makes the conversion time input dependant and the worst case
comparison time =~ $N^2/2*\tau*ln(2)$ is a two fold improvement in comparison time. But power saving may be significantly higher if actives are powered down after conversion and negligible if all power is dynamic.
For a latch $\tau = Gm/C = \beta*\sqrt(w)/C_{oxl}*W +C_{load}$ . if offset is non critical (SAR) or calibrated (multibit) no optimization is possible in this
end. At higher precision ($V_\delta~=10mV~=8bit$) noise (thermal/capacitive coupling noise $\propto \frac{1}{C}$) from this may become critical and a switch from a noisy to low noise
latch may be required.This will increase W and hence $\tau$.
\paragraph{}
	A Way to improve the comparison time from $O(N^2)$ to $O(N)$ will be to add gain stages in front of the Clocked comparator to the tune of $2^k$ for $k^{th}$ comparison . This will make compartison time 
constant for all clocks making comparison time $N*\tau*ln(2)$ . But this will add to settling portion of converision time and will be a trade off . This will automatically improve noise performance w.r.t clocked comparator.
\paragraph{}
Another option is to add a pipelined clocked comparator to the first one but not wait for it . So output of second latch will be $V_{in}*e^{(t1+t2)/\tau}$ making its resolution better 

\newpage
\subsection*{Non binary SAR}
\paragraph{}
	In a SAR ADC without any redundancy $k_{th}$ DAC step S(k) needs to settle to $1LSB = VREF/2^{N}$ for ADC to 
to converge correctly . In a constant clock environment this requires a time $T_{dac}=\tau*N*ln(2)$ corresponding to worst case first step and hence a total
DAC time of $\tau*N^{2}*ln(2)$. If each DAC step gets a settling time just enough to meet the settling requirement then $T_{dac}(k)=\tau*(N-k)*ln(2)$ and total dac
time will be $\tau*N(N-1)/2*ln(2)$. To make $T_{DAC}$ a constant we need redundancy propotional to current step so that an error $S(k)*e^{-T_{dac}/\tau}$ can be tolerated
.For this 
\begin{align*}
\sum_{(k+1)}^N{S(i)} &= (1+m)*S(k)\\
\sum_{(k+2)}^N{S(i)} &= (1+m)*S(k+1)\\
S(k+1) &= (1+m)*(S(k)-S(k+1))\\
S(k+1) &= (1+m)/(2+m)*S(k)\\
\end{align*}
This Points to a non binary radix $r=(1+m)/(2+m)$ . since redundancy greater than a step is not required $m<1$ and $r<2/3$. Total conversion clock = $\tau*ln(1/m)*ln(2)*ln(m+1)/ln(2+m)$.
Though this points to a near zero conversion time with m=1 its a mathematical anomally because of approximating error as $S(k)*e^{-T_{dac}/\tau}$ . More correct Derivation
is as follows (assumes non binary sar)
\begin{align*}
settling error &= \sum_1^k{S(i)*e^{(-(k-i)*T_{clk}+T_{dac})/\tau}}\\
S(k)&=1/2*r^{(k-1)}\\
settling error &= 1/2*r^{(k-1)}*e^{-T_{dac})/\tau}*\sum_0^{k-1}{r^-i*e^{-iT_{clk}/\tau}}\\
settling error &= 1/2*r^{(k-1)}*e^{-T_{dac})/\tau}*(1-r^-k*e^{-k*T_{clk}/\tau})/(1-r^{-1}*e^{-T_{clk}/\tau})\\
e^{-T_{clk}/\tau} &< r\\
settling error &~= 1/2*r(k-1)*e^{(-T_{clk}/\tau)}/(1-r^{-1}**e^{-T_{clk}/\tau}) = m*1/2*r^{(k-1)}\\
e^{-T_{dac}/\tau} = m/(1+rm) < r\\
r&~=1/2\\
T_{dac} &~= 2*ln(2)*\tau;
\end{align*}
}
{
	In a constant clock binary asynchronous ADC 's advantageous to constraint $$T_{cmp}(k)+T_{dacNamp}(k+1) = T_{Clk}$$ 
This is because if ${(k+1)}^{th}$ decision is critical $k^{th}$ decision is non critical and should take less time which can be used to make ${(k+1)}^{th}$
decision more precise.Also ${(k+1)}^{th}$ decision will take more time but since ${(k+2)}^{th}$ deciison is noncritical its dac settling error doesnt matter.
}
{

\end{document}
